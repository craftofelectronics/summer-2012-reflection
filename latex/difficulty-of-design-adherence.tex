\section{Difficulty of Design Adherence}

\begin{comment}
  This is all too chronological. Actual writing will need to condense a great deal of this so that we focus on the result---the fact that the compelling context is what made it easy to adhere to our design goals. When we didn't have clear design goals, it was easy to slip into old ways of thinking... but a clear context for the design to be expressed made it possible to put craft first as opposed to focusing on theory.
\end{comment}

* In the beginning: ``What does 'craft' mean?''

** Examples of craft from the WWW

** Examples of ``cool things'' using Arduinos

** Brainstorming of ideas/concepts required to do those things

* Wrestling with craft

** Initial categorization from brainstorming (inconclusive) (drafting lab)

** Toured craft locations in Berea

*** Dulcimer making

*** Berea Crafts manufacturing facility

*** Student craft store

* Revisiting Categorization (library day)

** Rethought categorizations

** Looked for connections

*** Relegated ``theory'' as a binder for Safety and Design

** Began to consider connections, not just categories

* Framing learning objectives

** Spent time discussing learning outcomes and possible activities

*** Came up with far too much.

* Detailed curricular ``fill-in''

** Breaking down semester into days/sub-days

** Exploring themes from learning objectives (Quantified Self)

** Emergence of model home

* Second pass ``fill-in''

** First week ``worked,'' a bit overfull.

** Second week disaster:

*** Weekend reading: Ohm's Law

*** First lab: see, it's Ohm's Law!

** Realization that we have just engaged in ``theory first'' curricular design.

** Not clear that initial fill-in was ``theory first,'' but detailed ``fill-in'' led directly to a theory-first mode of thinking.

** Hard to see how to break out of it.

* Design Flip

** Decided to start with the end product, try and move craft forward.

** Realized that context mattered supremely... moved house all the way forward.

** Enabled eas(y|ier) craft-first and hands-on design adherence. 

** Remain in context while studying sensors, electronics

