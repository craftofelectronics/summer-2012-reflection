%Design Local, Deliver Open

\begin{comment} % Notes from initial etherpad
* distributed / open source design is hard
* what does it mean for the course to be open?

* asked what would it mean for a course to be open; decided that we'd be designing for everyone and chose to focus on particular context
* developed personas & worked on understanding values in the context
* realized that distributed & open (course) design is incredibly hard
\end{comment}

\section{What does ``open'' mean?}

* FOSS
** OSS/Libre/CC

\section{``Open'' education}

* Open education (in general)
** What's easy
*** Kahn (if you want drill-and-practice)
*** MITx (people use it, do problems)
** What's hard
*** Apprenticeship / Mentoring
*** Hardware (see Open University new course)

\section{What does it mean in context?}

* What did it mean to us?
** Sharing materials?
** Student/community interaction
*** Notional open-hardware interactions
** Distance students?

\section{The Berea Context}

* Student population
* TAD / CS
** Required course
** ``Rebooting'' of course, new curriculum coming out of self-study
** New cross-disciplinary hire
* History of craft at Berea

\section{The Berea Contrast House}

* Nascent idea of model home
** Waived hands
* Concrete realization
** Affordable model developer/provider
** ``A phone call''
* A contextual home
** The Contrast House

\section{Importance of Artifacts}

* Context drives design (CONTEXT IS CRITICAL?)
* Artifacts enable collaboration
** Design processes are not ``open''; design products are open
** House is a rallying point for openness and collaboration
*** (Between us and the MakeCNC)
* Products beget products
** Model house + arduino + curricular materials become an ``educational kit''
** Wiring, videos, questions, etc. all provide guidance for students
** Students can use educational materials (in context)
** Contributors can develop more materials (in context)
